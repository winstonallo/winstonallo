\documentclass[10pt, letterpaper]{article}

\usepackage[
    ignoreheadfoot,
    top=1 cm,
    bottom=2 cm,
    left=2 cm,
    right=2 cm,
    footskip=1.0 cm,
]{geometry}
\usepackage{titlesec}
\usepackage[dvipsnames]{xcolor}
\definecolor{primaryColor}{RGB}{0, 0, 0}
\usepackage{enumitem}
\usepackage[
    pdftitle={Arthur Bied-Charretons Lebenslauf},
    pdfauthor={Arthur Bied-Charreton},
    pdfcreator={LaTeX with RenderCV},
    colorlinks=true,
    urlcolor=primaryColor
]{hyperref}
\usepackage[pscoord]{eso-pic}
\usepackage{calc}
\usepackage{changepage}
\usepackage{paracol}
\usepackage{needspace}
\usepackage{iftex}

\ifPDFTeX
    \input{glyphtounicode}
    \pdfgentounicode=1
    \usepackage[T1]{fontenc}
    \usepackage[utf8]{inputenc}
    \usepackage{lmodern}
\fi

\usepackage{charter}

\raggedright
\AtBeginEnvironment{adjustwidth}{\partopsep0pt}
\pagestyle{empty}
\setcounter{secnumdepth}{0}
\setlength{\parindent}{0pt}
\setlength{\topskip}{0pt}
\setlength{\columnsep}{0.15cm}
\pagenumbering{gobble}

\titleformat{\section}{\needspace{4\baselineskip}\bfseries\large}{}{0pt}{}[\vspace{1pt}\titlerule]

\titlespacing{\section}{
    -1pt
}{
    0.3 cm
}{
    0.2 cm
}

\renewcommand\labelitemi{$\vcenter{\hbox{\small$\bullet$}}$} % custom bullet points
\newenvironment{highlights}{
    \begin{itemize}[
        topsep=0.10 cm,
        parsep=0.10 cm,
        partopsep=0pt,
        itemsep=0pt,
        leftmargin=0 cm + 10pt
    ]
}{
    \end{itemize}
}


\newenvironment{highlightsforbulletentries}{
    \begin{itemize}[
        topsep=0.10 cm,
        parsep=0.10 cm,
        partopsep=0pt,
        itemsep=0pt,
        leftmargin=10pt
    ]
}{
    \end{itemize}
}

\newenvironment{onecolentry}{
    \begin{adjustwidth}{
        0 cm + 0.00001 cm
    }{
        0 cm + 0.00001 cm
    }
}{
    \end{adjustwidth}
}

\newenvironment{twocolentry}[2][]{
    \onecolentry
    \def\secondColumn{#2}
    \setcolumnwidth{\fill, 4.5 cm}
    \begin{paracol}{2}
}{
    \switchcolumn \raggedleft \secondColumn
    \end{paracol}
    \endonecolentry
}

\newenvironment{header}{
    \setlength{\topsep}{0pt}\par\kern\topsep\centering\linespread{1.5}
}{
    \par\kern\topsep
}

\newcommand{\placelastupdatedtext}{
  \AddToShipoutPictureFG*{
    \put(
        \LenToUnit{\paperwidth-2 cm-0 cm+0.05cm},
        \LenToUnit{\paperheight-1.0 cm}
    ){\vtop{{\null}\makebox[0pt][c]{
        \small\color{gray}\textit{Zuletzt aktualisiert im Juli 2025}\hspace{\widthof{Zuletzt aktualisiert im Juli 2025}}
    }}}
  }
}

\let\hrefWithoutArrow\href

\begin{document}
    \newcommand{\AND}{\unskip
        \cleaders\copy\ANDbox\hskip\wd\ANDbox
        \ignorespaces
    }
    \newsavebox\ANDbox
    \sbox\ANDbox{$|$}

    \begin{header}
        \fontsize{25 pt}{25 pt}\selectfont Arthur Bied-Charreton

        \vspace{5 pt}

        \normalsize
        \mbox{Wien}%
        \kern 5.0 pt%
        \AND%
        \kern 5.0 pt%
        \mbox{\hrefWithoutArrow{mailto:arthurbiedchar@gmail.com}{arthurbiedchar@gmail.com}}%
        \kern 5.0 pt%
        \AND%
        \kern 5.0 pt%
        \mbox{\hrefWithoutArrow{tel:+436641502349}{+43 (0) 664 1502349}}%
        \kern 5.0 pt%
        \AND%
        \kern 5.0 pt%
        \mbox{\hrefWithoutArrow{https://linkedin.com/in/arthur-bied-charreton}{LinkedIn}}%
        \kern 5.0 pt%
        \AND%
        \kern 5.0 pt%
        \mbox{\hrefWithoutArrow{https://github.com/winstonallo}{GitHub}}%
    \end{header}

    \vspace{15 pt - 0.3 cm}

    \begin{onecolentry}
        Softwareentwickler mit Leidenschaft für Systemprogrammierung, Infrastruktur und Open Source.
    \end{onecolentry}

    \vspace{-0.1 cm}

    \section{Ausbildung}

    \begin{twocolentry}
    {Voraussichtlich 2026}
        \href{https://www.42network.org/blog/breaking-news-42-ranks-3rd-worldwide-for-innovation/}{\textbf{42}}, Software Engineering (\href{https://europass.europa.eu/en/description-eight-eqf-levels}{EQR 7}, Master-Niveau)
    \end{twocolentry}

    \vspace{0.10 cm}

    \begin{onecolentry}
        \begin{highlights}
            \item \textbf{Note:} Top 1\%
            \item \textbf{Kursinhalte:} Betriebssysteme, Netzwerktechnik, Systemprogrammierung, DSA, OOP, DevOps, Kryptographie
            \item \textbf{Tutoring:} Betreuung und Bewertung von Kandidaten während des hochselektiven Auswahlprozesses
            \item \textbf{Mentoring:} Wöchentliche Einzelbetreuung von Studenten mit detaillierten Code-Reviews
        \end{highlights}
    \end{onecolentry}

    \section{Berufserfahrung}
        
    \begin{twocolentry}
        {März 2025 – Aug 2025}
        \textbf{Software Engineer (Praktikum)}, Fronius -- Wien, Österreich
    \end{twocolentry}

    \vspace{0.10 cm}
    
    \begin{onecolentry}
        \begin{highlights}
            \item Entwurf und Implementierung eines Validierungssystems für ML-Pipelines, das In/Outputs jedes Schritts überwacht und frühzeitig Anomalien erkennt
            \item Entwicklung eines Python-SDKs zum Mocking von Pipeline-Abhängigkeiten in Integrationstests, Anomalieerkennung von manuellen Tests ganzer Pipelines zu automatisierten CI-Tests einzelner Schritte verlagert
            \item Kontinuierliche Verbesserungen am CI-Setup, Ausführungszeit im Durchschnitt um über 80\% reduziert
        \end{highlights}
    \end{onecolentry}

    \vspace{0.2 cm}

    \begin{twocolentry}
        {Juli 2024 – Januar 2025}
        \textbf{Software Engineer (Praktikum)}, EVVA -- Wien, Österreich
    \end{twocolentry}

    \vspace{0.10 cm}
    
    \begin{onecolentry}
        \begin{highlights}
            \item Entwicklung eines E-Mail-Klassifizierungssystems in Rust und Python, das die manuelle Bestellabwicklung um 2 Vollzeitst-Äquivalente reduzierte
            \item Bereitstellung einer lokalen CI/CD-Pipeline, wodurch die Deployment-Zeiten erheblich verkürzt wurden
        \end{highlights}
    \end{onecolentry}

    \vspace{0.2 cm}

    \begin{twocolentry}
        {Februar 2024 – Juli 2024}
        \textbf{Student Support - IT Delivery}, Raiffeisenlandesbank NÖ-Wien -- Wien, Österreich
    \end{twocolentry}

    \vspace{0.10 cm}

    \begin{onecolentry}
        \begin{highlights}
            \item Unterstützung bei der Umstellung von Excel auf Jira/Confluence für das Projektmanagement
            \item Zusammenarbeit mit Teams zur Optimierung des Issue-Tracking-Prozesses
        \end{highlights}
    \end{onecolentry}

    \section{Projekte}
        
    \begin{twocolentry}
        {Rust, x86 Assembly}
        \href{https://github.com/kentucky-fried-kernel/kfs}{\textbf{32-bit Kernel}}
    \end{twocolentry}

    \vspace{0.10 cm}
    
    \begin{onecolentry}
        \begin{highlights}
            \item Entwicklung eines bootfähigen 32-bit Kernels
            \item Implementierung von Kernfunktionalitäten wie ASM-Boot-Code, Bildschirmschnittstelle und Interrupt-Tabellen
        \end{highlights}
    \end{onecolentry}


    \vspace{0.2 cm}

    \begin{twocolentry}
        {C++}
        \href{https://github.com/winstonallo/webserv}{\textbf{HTTP Webserver}}
    \end{twocolentry}

    \vspace{0.10 cm}
    
    \begin{onecolentry}
        \begin{highlights}
            \item Entwicklung eines Nginx-inspirierten HTTP-Webservers, mit Support für GET, POST, PUT und DELETE
            \item Vollständig single-threaded, basiert ausschließlich auf non-blocking I/O für die Anfrageverarbeitung
            \item Unterstützung für Cookies, CGI, Datei-Uploads und statische Inhaltsbereitstellung
        \end{highlights}
    \end{onecolentry}

    \vspace{0.2 cm}

    \begin{twocolentry}
        {Rust}
        \href{https://github.com/winstonallo/taskmaster}{\textbf{Unix Job Control System}}
    \end{twocolentry}

    \vspace{0.10 cm}
    
    \begin{onecolentry}
        \begin{highlights}
            \item Verwaltet Prozesse mit konfigurierbaren Richtlinien, Healthchecks und einer Controll-Shell
            \item Verwendet eine State Machine zur Verfolgung prozessbezogener Ereignisse
            \item Support für Privilege Deescalation, Konfiguration Hot-Reload und Echtzeitüberwachung
        \end{highlights}
    \end{onecolentry}

    \vspace{0.2cm}

    \begin{twocolentry}
        {C, x86 Assembly}
        \href{https://github.com/winstonallo/woody-woodpacker}{\textbf{64-bit ELF Packer}}
    \end{twocolentry}

    \vspace{0.10 cm}

    \begin{onecolentry}
        \begin{highlights}
            \item Entwicklung eines Packers für 64-bit ELF-Binaries mit PIE-Support
            \item Pack-time Code-Verschlüsselung und Self-Decryption via Shellcode-Injection
        \end{highlights}
    \end{onecolentry}

    \section{Fähigkeiten \& Technologien}

    \begin{onecolentry}
        \textbf{Programmiersprachen:} Rust, Python, C, C++, Golang, TypeScript, SQL, Bash
    \end{onecolentry}

    \vspace{0.2 cm}

    \begin{onecolentry}
        \textbf{Technologien:} Git, Linux, Docker, Kubernetes, GitLab CI/CD, GitHub Actions, Jenkins, Ansible
    \end{onecolentry}

    \vspace{0.2 cm}

    \begin{onecolentry}
        \textbf{Sprachen:} Französisch (Muttersprache), Deutsch (Fließend), Englisch (Fließend)
    \end{onecolentry}

\end{document}
